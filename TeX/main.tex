\documentclass[12pt]{report}
\usepackage{cite}
\usepackage{amsmath,amssymb,amsfonts}
\usepackage{algorithmic}
\usepackage{graphicx}
\usepackage{textcomp}
\usepackage{xcolor}
\usepackage{booktabs}
\usepackage{tabularx}
\usepackage{hyperref}

%KUstyle
% \usepackage{KUstyle}
% \ptype{Social Data Science}
% \subtitle{An Attachment-Based Therapy Metric}


% Citation style packages
% \usepackage{apacite} % Package for APA citations
% \bibliographystyle{apacite}
\bibliographystyle{plain}

% This change the content of the frontpage
\title{Anxious or Avoidant? Securing Reliable Repeated Measures of Adult Attachment Through Machine Learning}
\author{Frederik Bredgaard}
\date{May 31st 2024}

\renewcommand{\contentsname}{Table of content}

\begin{document}

\maketitle
\
\tableofcontents

\chapter{Background and Theory}
\section{Introduction}
While the development of attachment theory has contiously delivered new and valuable insights into the development of children and the relationships, pathologies, and treatments of adults through the last six decades since its inception, its clinical application has remained mostly theoretical until somewhat recently. Arguably, this is not for a lack of clinical relevance but rather due, in large part, to the cumbersome measurement of the constructs belonging to the theory.
As will be covered below, several instruments and methods for the assessment of attachment style exists. However, the gold standard for assessing adult attachment, the Adult Attachment Interview (AAI) \cite{AAITest}, is a time-consuming measure, making its application in large-scale research and clinical settings rather limited. The objective of this thesis is to develop on existing methods such as the Patient Attachment Coding System \cite{Talia2017}, adding a degree of automation through language modelling approaches derived from machine learning. While the available data is limited at this stage of the field, I believe that future work can build on the approach developed here to produce statistical models for autocoding the AAI as well as a clinically relevant tool that can assist clinicians and researchers alike by making the assessment of attachment in adults more easily available and significantly more scalable.

As such, I will investigate the feasibility and utility of automatically assessing psychotherapy patients' attachment characteristics.

The clinical utility of the approach is assessed first, through a review of the link between theory and empirical data, including implications on health, happiness, and well-being.

Second, the feasibility of automatically classifying a patient's attachment style is investigated through a series of language modelling experiments, based on the limited available data.


\section{Attachment Theory}
This section outlines the most central elements of attachment theory from its development to the clinical relevance of different patterns of attachment and how the theory more broadly explains or mediates the effects of psychotherapy.

From this theoretical perspective, an attachment may be understood as an affectional tie formed between an individual and some other and which is characteristed by behaviours seeking to gain and maintain proximity to the other. This tie and its associated behaviours bind the two individuals together in space and endures over time but the proximity-seeking behaviours are particularly prevalent during times of distress \cite{Ainsworth1970,Bowlby1988}.

\subsection{Theoretical Development}
The theory of attachment is a fundamentally ethological approach, which at the time of its development sought to explain behaviours that were poorly accounted for in existing theories. The initial development of the theory is often credited to John Bowlby, who originally formulated it from a psychoanalytic, Freudian perspective. However, the framework would soon develop through a more empirically informed approach, not least thanks to the groundbreaking work on infant attachment done by Mary Ainsworth and her colleagues \cite{Ainsworth1970}.

Bowlby's arrival at the tenets of attachment theory was influenced by the psychoanalytic exploration of object relations - a particular area of Freudian thought concerned with ego development and the relation of the psyche to external objects and persons. Bowlby, however, viewed the apparent attachment behaviours as being mostly adaptive to external needs and stimuli rather than as responses to internal fantasies. In line with the tradition that inspired him, Bowlby based his initial theories on case studies, particularly of delinquent children (e.g. \cite{bowlby1946thieves}). From these cases, Bowlby and his colleagues found that children displaying criminal or otherwise problematic behaviours or mental or emotional distress typically struggled to form meaningful relationships and many had been repeatedly institutionalised or moved between foster homes \cite{bowlby1951WHO}.

While the foundation for the theory and its conclusions had been laid long before, the first widely influential published piece on the effects of childhood care and attachment on mental health was Bowlby's 1951 monograph published by the World Health Organization under the title \textit{Maternal Care and Mental Health} \cite{bowlby1951WHO}. The central assertion of this work, which reviewed the limited existing empirical data, was that healthy infant development presumes a warm and continuous relationship with the mother. At the time, this proposition was controversial as it broke with traditional views on child rearing and development. Further, the supporting evidence was limited and neither psychology nor medicine offered any coherent well-developed theoretical explanations for the mechanisms leading from maternal care to mental health \cite{Bowlby1988, who1962deprivation}. Nevertheless, the publication sucessfully brought attention to the importance of maternal care for developing children.

Following the critique of his 1951 monograph, Bowlby 

\subsection{Current View of Attachment and Its Significance for Health and Daily Life}
While attachment is measured with a variety of methods with varying degrees of granularity today, the methods tend to agree on the underlying structure of attachment based on the continnuous orthogonal dimensions of anxiety and avoidance, allowing the classification into distinct styles. This structure can be thought of as describing the main tenets of attachment theory as it is understood and applied today. This section offers a non-exhaustive review of relevant links between theory, practice, and observable outcomes. This relies on understanding the attachment styles measured using different approaches as essentially analogous or interchangeable even if they are presented with different names, as is custom in meta-analyses combining data from different measures (e.g., \cite{McConnell2011,Pinquart2013}).

\subsubsection{Stability and Change}
To understand predictive ability of attachment style, which constitutes the most direct link between theory and evidence, the assumption of stability must be addressed. Generally, an individual's state of mind with regards to attachment is theorised to be relatively stable throughout the lifespan. This theoretical assumption is based primarily on Bowlby's own speculation. Namely, in his influential book \textit{Attachment and Loss}, first published in 1969, Bowlby theorises that expectations regarding attachment figures are acquired in infancy and early childhood and thus should remain stable throughout adulthood \cite{bowlby1982attachment}. However, despite the stability of attachment styles being an axiom underpinning the theory as a whole, it has proven persistently difficult to confirm or reject the assertion empirically.

In the empirical literature, attachement classifications tend to show both stability and fluidity, with studies involving younger individuals finding only somewhat stable patterns with a greater degree of fluidity and studies of older individuals finding much greater stability.

Broadly, the available evidence points to mostly stable patterns of attachment over long periods of time. This is perhaps most convincingly demonstrated by a 2013 meta-analysis of 127 studies on attachment stability with a total \textit{N} of 21,072 by Pinquart, Fueßner, and Ahnert \cite{Pinquart2013}. Here, the authors found medium-sized stability with a test-retest coefficient of .39. However, while no significant stability was detectable in studies running longer than 15 years, coefficients were larger for shorter time intervals and for samples not comprised of at-risk children.
This suggests that the applied attachment measures have good reliability, and that while attachment displays good stability for shorter time intervals, life events, circumstances, and perhaps deliberate interventions can change it over longer intervals.

Similiarly, in a study by Zhang and Labouvie-Vief \cite{Zhang2004}, a sample of 370 individuals ranging in age form 15 to 87 was assessed regarding attachment, well-being, coping strategies, and depressive symptoms three times over a six year period. Attachment style was found to be reasonably stable although changes were observed. Test-rest reliability in this sample was between .40 and .49 between the first and second assessment and between .24 and .45 between the first and third measurement. This could indicate that attachment style does change, although this change is most likely to occur over a period of more than two years. For comparison, Zhang and Labouvie-Vief cite studies demonstrating that test-retest reliability of big five personality traits over a similar six year period tend to be around .60 - .80 after age 30 \cite{Costa1988,Roberts2000}.

The changes observed are themselves relevant to our understanding of the development, stability, and significance of attachment. Namely, Zhang and Labouvie-Vief found that change towards greater attachment insecurity was associated with defensive coping strategies characterised by rigid, immature and maladaptive ways of interacting with the world. Depressive symptoms was also a significant predictor of change towards greater insecurity. In contrast, change towards greater security was predicted by flexible and reality-oriented coping strategies and better perceived well-being.
Finally, Zhang and Labouvie-Vief found an effect of age effect on the direction of the observed changes. Specifically, it appears that over time, adults may become more secure and more dismissive but less preoccupied  \cite{Zhang2004}.

Lending further support to the notion of increased attachment stability with age, Consedine and Magai assessed attachement in 415 older adults at age 72 and again at age 78. In this sample, more than 80 \% of participants remained stable in their classification.

Studies into the hierarchies of attachment also further our understanding of the quality of the development of attachment through the lifespan.
Based on a study of mid- and late-adolescents recruited from a highschool and a university, respectively, Rowe and Carnelley \cite{Rowe2005} concluded that peers become increasingly important attachment figures as people age. While the undergraduate students assessed did not rate their parents as any less important to their core self than the highschoolers did, they considered their friends significantly more central to their core self.
This and other studies (e.g. \cite{Fraley1997,Doherty2004}) paint the picture of adolescents expanding their circle of close attachments to include their peers, but that this expansion does not come at the expense of the strength of attachment to parents or primary caregivers.
As adolescents become adults, they tend to show a decline in how highly they rate their parents as attachment figures and how much they rely on them for attachment-related needs. In their place, there is strong support for the notion that adults tend to rely more on friends and romantic partners as they age and in particular as their peer- and romantic relationships go longer \cite{Tancredy2006,Doherty2004,Fraley1997}.

It is generally accepted that changes to attachment style are common following major life events such as loss, change in relationship status, or becoming a parent. However, the specific causes of changes to attachment style are empirically not well understood. While Kirkpatrick and Hazan \cite{Kirkpatrick1994} found relationship initiation or break-up to be a mediator of attachment change, many studies are unable to attribute any observed changes directly to life events.
One such study is an 8-month examination of 144 young adults by Scharfe and Bartholomew \cite{Scharfe1994}. They used several forms of assessment and overall found moderate stability. The observed stability was noticeably higher for expert ratings than self-report measures, but, independent of measurement method, Scharfe and Bartholomew found no consistent relationship between changes in attachment security and life events between the two measurement occasions.
Similiarly, Cozzarelli et al. \cite{Cozzarelli2003} did not find strong associations between a long list of life events and attachment changes over a two-year period in a sample of 442 women who underwent an abortion.

The best evidence on how attachment changes has come more recently. In a 2021 study, Fraley, Gillath, and Deboeck followed a sample of over 4,000 people assessed in multiple waves for between six and 40 months. They found that changes in attachment security occurred following life-events related to relationships, career, family, and more. However, most changes were transient and the majority of people would revert to their original attachment style given enough time after most life events.
Nevertheless, some events did tend to lead to enduring changes, suggesting that some experiences are likely to affect change in attachment style.
The greatest enduring effects on increasing general attachment anxiety occurred following such events as entering retirement, receiving a work-related promotion, starting school or university, or moving to a new location. A shared characteristic of these events is a great change in social network. Following such life-changing events, individuals likely enter new communities, perhaps leaving others, and an increased anxiety around relationships may follow from spending more time with new, less close, relationships which inherently feel less secure. Simultaneously, spending less time in one's existing close relationships may weaken these relations or one's certainty of them,, effectively making them less secure.
Notably, Fraley et al. only found one event which had a significant lasting effect on \textit{decreasing} general attachment anxiety. This event was finding out that oneself or one's partner was pregnant, which had the second-largest enduring effect on general attachment anxiety, only exceeded in magnitude by the opposite effect of retiring from work.
Compared to the anxiety measure, avoidance was more stable in this sample. In fact, Fraley et al. only found two events to be associated with significant enduring changes in general attachment avoidance. These were getting married, which was associated with increased avoidance, and oneself or one's partner giving birth, which was associated with decreased avoidance.
Importantly, there were significant individual differences in the extent to which people changed. In line with previous research (e.g. \cite{Zhang2004}), positive or negative appraisals of the given experience were related to the extent of attachment changes on the individual level.

Finally, the concept of volitional change to one's attachment style is important, yet poorly described empirically. Similiarly to the research on the impact of life events, we have only recently seen strong direct evidence of deliberate changes to attachment security. This came in 2020 when Hudson, Chopik, and Briley \cite{Hudson2020} conducted two studies to construct and validate a measure of people's desire to change attachment characteristics followed by a 16-wave weekly longitudinal study totalling more than 4,000 participants combined.
In the first two studies, Hudson et al. found significant individual differences in desire to change attachment-related attributes. Crucially, these differences were related to measured trait levels of attachment anxiety and avoidance and to satisfaction with relevant life domains. This followed the theoretically expected pattern that people with greater dissatisfaction in relevant domains as well as people with higher measured levels of anxiety or avoidance were more likely to want to change those traits.
Finally, the 16–wave weekly longitudinal study showed that desire to change predicted observed change not only at the level of security-insecurity but in the expected domain such that people who wanted to become less anxious generally experienced less attachment anxiety over time and people wishing to decrease their avoidance generally did so. Impressively these effects remain significant when controlling for relationship status changes.


In summary, it appears that attachment is a relatively stable construct of individual differences which, apart from transient state-like changes following some major life events, changes and develops at a pace measured on the scale of years. Nevertheless, it is not clear exactly how or why changes to attachment style occur, and McConnel \cite{McConnell2011} points out that the factors influencing attachment in adulthood are complex, ranging from internal and behavioural factors such as coping and well-being to external factors like life events and environmental stress. This may make attachment more malleable through deliberately intervention in adulthood and more difficult to predict over very long periods as is shown by e.g., Pinquart et al. \cite{Pinquart2013}.
Further, individual differences and psychological factors, including coping strategies \cite{Zhang2004} and appraisals \cite{Fraley2021}, are important in mediating the effects of life events. The importance of psychological factors provides some support for the notion of volitional changes as has been convincingly demonstrated by Hudson et al \cite{Hudson2020}.
Based on the available evidence, one must conclude that attachment styles are somewhat stable but that they change naturally throughout the life cycle \cite{Rowe2005,Doherty2004,Fraley1997} and in response to both life events \cite{Fraley2021} and deliberate interventions \cite{Hudson2020}.

With the relative stability and some understanding of the changing of attachment styles established, I will now turn to the importance of attachment security in a wider context. As such, the next sections cover the predictive value of the theory and its constructs in regards to relationships, daily life, health, psychopathology, and mental health. This is not meant to be an exhaustive review but rather I aim to elucidate the remarkable extent to which attachment permeates pivotal facets of human experience. Consequently it will serve to illustrate the value of the construct as pliable, accessible, and informative variable for patients and clinicians alike.

\subsubsection{Relationships and Daily Life}
Mental health goes beyond the absence of psychiatric illness or mental distress and great deal of influence on relationships and daily life is associated with attachment classifications. Some of this influence is briefly covered here.
Concerning romantic relationships, attachment insecurity is associated with lower levels of relationship evaluations even when controlling for important factors such as gender roles, romantic beliefs, and self-esteem \cite{Rodriguez2021, Jones1996}.
Further, in promoting relationship satisfaction and security for anxious and avoidant individuals in romantic relationships, different strategies employed by their partners have different effects depending on attachment style \cite{Overall2015}. The insights gained from research into regulation of attachment insecurity in romantic relationships has also been suggested as a on guide how clinicians should approach patients with different attachment patterns to help them distinguish the best strategies for treatment and building a strong therapeutic alliance, as is covered in section \ref{sec:Differentiable approaches}.

It is also clear that attachment styles influence how individuals process events and attachment cues in their relationships. This may lead to biased expectations and perceptions of their relationships and partners and eventually interfere with well-functioning relationship processes \cite{Collins2007, Collins2004, Hazan1994, Mikulincer2003, Rodriguez2019}.
For instance, insecurely attached partners in romantic relationships tend to react in a counterproductive manner when faced with interdependence dilemmas in their romantic relations \cite{Simpson2012}.

Lastly, concerning everyday distress and social functioning, Sheinbaum et al. \cite{Sheinbaum2015} followed 206 you adults using an experience-sampling methodology.
They found that cognitive appraisals, social functioning, and moment-to-moment affective states were predicted by attachment style.
Namely, individuals with anxious attachment experienced higher negative affect, stress and perceived more social rejection in their day-to-day experience and interactions. Individuals with avoidant attachment, however, seemed to experience greater degree of deactivation such as decreased positive states and lowered desire for social interaction when alone.
Additionally, individuals with anxious attachment experienced social situations with low levels of rated "closeness" significantly more negatively than those with secure attachment \cite{Sheinbaum2015}.

\subsubsection{Health, Psychopathology, and Mental Health}
Attachement theory and its supporting evidence have serious implications for general healthcare with associations reaching from the symptoms experienced and reported to the tendency to visit health professionals.
In a survey-based study of 287 university students assessed twice ten weeks apart by Feeney and Ryan \cite{Feeney1994}, anxious attachement was linked to higher symptom-reporting. This association remained robust even when controlling for general negative emotionality, suggesting that those with anxious attachement may be more prone to experience or report signs of distress.
The same study also found that avoidant attachment was associated with fewer visits to health professionals, even when symptom reports were controlled for, suggesting that avoidant attachment is associated with delayed seeking of medical care \cite{Feeney1994}.

The occurence and development of psychological distress may also be associated with attachment insecurity. A 2017 review by Dagan, Facompré, and Bernard \cite{Dagan2018} of 55 studies assessing a totalt of 4,386 participants using the AAI and well-validated depression measures found significant associations between attachment classification and depressive symptoms.
Namely, insecure attachment was associated with depressive symptoms. This effect seems to be carried primarily by the preoccupied participants as the dismissing participants alone did not exhibit significantly more depressive symptoms.

The notion that attachment insecurity is associated with clinically relevant psychological distress is supported by a large study of 5,645 participants by Meng, D'Arcy, and Adams \cite{Meng2015}.
In this study, the authors compared the use of various healthcare services across attachment classifications. They found that individuals who reported insecure attachment styles were significantly more likely to have used a variety of healthcare services including having had a session of psychotherapy or a prescription for pharmacological treatment of mental or behavioural problem. This was the case for both anxious and avoidant insecurity, while only individuals with anxious attachment were more likely to report having used online support groups or forums.

The phenomenology and symptomatology of psychopathology may also differ between individuals according to their differences on the attachment scales.
This was demonstrated by comprehensive study of 500 patients with psychosis from three countries by Korver-Nieberg et al. \cite{Korver-Nieberg2015} which found that attachment anxiety (but not avoidance) predicted the severity of both positive and affective symptoms while both anxiety and avoidance were associated with severity of hallucinations and persecution.
Similiarly, in a smaller sample of 40 persons with serious psychopathology, Dozier \cite{Dozier1990} found that attachment security was associated with affective symptomology rather than thought disorders.

Attachment orientation appears to influence therapy outcomes as well. As is described in section \ref{sec:Applying attachment in therapy}, this influence may be mediated by lower ratings of therapeutic alliance associated with insecure attachment \cite{Baier2020}, which in turn may be caused by difficulties with building and maintaining a strong alliance by resolving ruptures.
The importance of attachment for therapy outcomes is made apparent in a number of studies. Most notably, a review of three meta-analyses with a combined \textit{N} of 1,467 found a medium effect size of attachment anxiety on treatment outcome ($d=-.46$) while overall security had a smaller yet relevant effect size ($d=.37$) \cite{Levy2011}. This adds further credibility to the notion that greater attachment anxiety is associated with poorer post-treatment outcomes, while the inverse is true for attachment security. Avoidance was not significantly correlated with therapy outcomes.


\section{The Measurement of Attachment}
Lorem

\subsection{The Strange Situation}
Lorem

\subsection{Adult Attachment Interview}
The adult attachment interview is a semistructured interview that asks respondents about autobiographical memories from early childhood related to attachment. The interviewer asks respondents to review and evaluate these experiences from their current adult perspectives. The interview is coded according to the ways in which these childhood experiences are described and the reflections offered by respondents. Thus, the coding process is less concerned with the content of the specific content of memories and more interested in the organisation of thoughts and memories. This cognitive organisation according to early relationships is categorised into four attachment styles roughly corresponding to those observed in children and in most other measures of attachment patterns. These styles are labelled secure, dismissing, preoccupied, and disorganised \cite{Hesse1999, AAITest}.

The measurement of attachment in adults is typically made using the de facto gold standard of attachment research, the Adult Attachment Interview (AAI) \cite{AAITest, Talia2019, haltigan2014adult}. The administration of the AAI takes between 60 and 90 minutes while the required verbatim transcription may take four to ten hours and coding of the transcript is expexted to take at least four hours by a coder requiring extensive training, prompting some researcher to search for simpler or less ressource-intensive methods of measurement \cite{Haas1994}.

\subsection{Patient Attachment Coding System}
Lorem ipsum \cite{Talia2017}

\section{Applying Attachment Theory in Therapy}
\label{sec:Applying attachment in therapy}
The specific ways in which attachment influences the process of therapy and the strategies that therapists employ to face the challenges of and leverage the opportunities offered by different attachment states of mind are complex. There is however, a foundation of theory and research into this and related concepts, some of which is covered in the following sections on applying attachment-related concepts to clinical practice. It should be noted that although the attachment security of therapists may be an important variable influencing the patient's experience and the therapist's effectiveness \cite{Mikulincer2013, Daniel2006, Dozier1994, Cologon2017, Talia2020}, it is not covered in this thesis.

The relevance of attachment, whether as a variable of individual difference to be accounted for or as in-session behaviours and dynamics to recognise and act on, is acknowledged by practitioners of almost all types of therapy. Across modalities, clinicians now view attachment dynamics as primary to the function of psychotherapy and many directly apply them to interpret their patients' experiences and to select and design interventions. According to Slade and Holmes \cite{Slade2019}, attachment-informed therapy offers three core principles that apply universally, regardless of a patient's attachment style. These principles remain valuable even before a formal assessment is conducted.

First, by recognising the basic attachment needs such as emotional safety and a secure base for exploration, therapists can more readily foster the environment and secure relationship needed for patients to address their challenges.

Second, the adaptation of therapeutic practice to specific needs and the therapist's ability to recognise attachment behaviours and reactions related to insecurity is a key competence characteristic of experienced practitioners. This clinical attention is relevant not only when the attachment style of the patient is known and has been assessed but for any patient as it provides valuable insight into in-session efforts of the patient to regulate their emotions.

Third, therapy can change attachment styles. In attachment-informed psychotherapy, the therapists aims to be a secure base for their clients to explore their experiences and emotions, thus fostering autonomy, openness, and an affectively coherent sense of self. Through the therapeutic process, clients can become more effective at regulating their anxiety in relation to attachments, as in relation to anything, and can learn to gradually change their behaviours and patterns of thought, over time leading to measurable change in attachment styles.
These changes occur even in clinical samples with very serious psychopathology \cite{Fonagy1996} and, in fact, patients with borderline personality disorder may be among those who can expect the greatest shift in the direction of security \cite{Levy2006, Stovall2003}.

\subsection{The Therapeutic Alliance}
While the therapeutic alliance is not a construct directly under attachment theory, their relevance to each other are apparent. The therapeutic or working alliance is a well-studied measure highly related to therapeutic outcomes and experiences and the theoretical and empirical links between attachment and alliance constructs are well established.

\subsubsection*{What is the therapeutic alliance and why does it matter?}
Early conceptualisations of the working alliance, such as that by Edward Bordin \cite{Bordin1979}, emphasized an alignment between client and therapist regarding the goals and objectives of therapy. However, Bordin also recognized the importance of an emotional connection built on trust and mutual respect. He proposed that the working alliance is comprised of three key components: goal, task, and bond. The bond component refers to the attachment-like relationship that develops between therapist and client.

The therapeutic alliance has emerged as stand-out measure in the vast literature on and concepts surrounding psychotherapy. Summers and Barber \cite{Summers2003} argued in 2003 that this is because it is shown to have robust effects on treatment outcomes across therapist traditions and patient conditions.
A more recent review corroborates this stance as a meta-analysis of four decades of research on the therapeutic allliance, covering 295 studies with a total \textit{N} of more than 30,000 patients by Flückinger et al. \cite{Fluckinger2018} concluded that the correlation between various measures of both the therapeutic alliance and patient outcomes is very robust.
This was the case irrespective of treatment approach, patient characteristics and whether the therapy was conducted face-to-face or online.

The robust nature of these associations has garnered significant research attention within psychotherapy. This research has led some scholars to posit that a strong working alliance may be the central mechanism or primary facilitator of therapeutic change \cite{RodgersCailholBuiEtAl2010}.
In a 2020 review of 37 studies and the extent to which they meet criteria for mechanistic research, Baier, Kline, and Feeny \cite{Baier2020} found that the therapeutic alliance mediated therapeutic outcomes in 70 \% of the studies included.
Therefore, the therapeutic alliance remains a crucial variable for understanding therapeutic change, even in studies where it is not directly assessed.  This underscores the importance of investigating the mechanisms by which the alliance exerts its influence on treatment outcomes as well as the mechanisms by which the therapeutic alliance is built and maintained.

\subsubsection*{Patient attachment and therapeutic alliance}
The link between attachment and therapeutic relationship may seem intuitive and indeed there is evidence to suggest that the patient's experience of the therapeutic relationship is influenced by their AAI classification \cite{Talia2019}.
This is supported by a 2014 meta-analysis by Bernecker, Levy, and Ellison \cite{Bernecker2014}, where weak but very robust associations were found between patient-rated working alliance and both attachment anxiety and avoidance. Both relationships were negative, such that higher avoidance or anxiety predicted worse alliance ratings.
Patient attachment to their therapist is strongly related to ratings of the strength of therapeutic alliance. In fact, Mallinckrodt, Gantt, and Coble \cite{Mallinckrodt1995} found correlations of around .80 between the two measures. This suggests that they are not only strongly related but may, at least in some instances, be indications of the same underlying construct.

Attachment security may influence the therapeutic alliance in more than one way.
A 2011 meta-analysis by Diener and Monroe \cite{Diener2011} of the association between attachment style and self- and therapist-rated therapeutic alliance found that greater attachment security predicted stronger therapeutic alliances. This relationship was significant for both types of ratings but significantly stronger for patient ratings than therapist ratings of the alliance.
No significant differences in the strength of alliances were found between anxious and avoidant attachment styles.
This could suggest that the therapist's own attachment characteristics play an appreciable role in the building and assessment of the alliance and perhaps that therapists should place lower confidence in their own ratings of the alliance when the patient has greater attachment insecurity.

It could also be the case that the alliance building process differs depending on the patient's attachment characteristics.
In fact, dismissing patients seems to be less sensitive to or invested in the working relationship with their therapist overall. Following 36 former political prisoners over 10-12 months of therapy, Kanninnen, Salo, and Punamäki \cite{Kanninen2000} found that patients' view of the therapeutic alliance developed differently depending on their attachment style.
While all three groups (secure, dismissing, preoccupied) rated their alliance similarly after the first session, dismissing patients were the only group to not experience a drop followed by an increase towards the end of their treatment. Rather, they rated their alliance similarly to the beginning in the middle of their treatment and then substantially worse at the end of treatment.
This may be related to findings by Eames and Roth \cite{Eames2000}, who followed 30 patients, tracking the working alliance and ruptures to it as reported by both patients and therapists. Here, as the theory might predict, preoccupied attachment was associated with more reports of ruptures and dismissing attachment with fewer.

It appears that dismissing patients are more stable in their evaluations of therapeutic relationships, but that they may require more reassurance or work specifically on the alliance later in the therapeutic process.
In contrast, preoccupied patients seem to hold more shifitng and unstable views of the relationship with their therapist, requiring more work throughout by paying close attention to situations that could be interpreted as alliance ruptures.
This could suggest that therapists should invest in interventions targeting the therapeutic alliance at different times throughout the course of treatment depending on their patient's attachment style.

It is clear that both patients with insecure attachment and their therapists rate the therapeutic alliance lower than is the case for patients with secure attachment.
However, the development through time of the alliance appears to be differentiated as well. Preoccupied patients may experience more ruptures to the alliance while dismissing patients are less sensitive in this regard.
This is valuable and directly actionable insights for therapists, who may want to adjust their therapeutic approach and consider not just how much but \textit{when} to invest in building the alliance depending on their patients' displayed attachment characteristics.

\subsection{Differentiable Approaches}
\label{sec:Differentiable approaches}
Beyond the established influence of in-session attachment dynamics and patient classification on the therapeutic alliance, this paper explores a more transformative and potent application of attachment theory in psychotherapy.
This approach involves tailoring therapeutic practice to the specific needs and developmental potential associated with the patient's unique attachment characteristics and state of mind.
Although this approach is not formalised in one universally accepted manual, I will review research showcasing its potential and demonstrating that experienced and skilled therapists already employ this type of tailoring in their practice.

Part of the basis for this approach comes from the observation that patients behave very differently in therapy and that they do so in ways partly predictable from their attachment characteristics. This was shown in an exploratory study by Daniel \cite{Daniel2011} who analysed patients' speech patterns in both psychodynamic and cognitive-behavioural therapy.
Notable differences were that preoccupied patients talked more and had longer speech turns than their dismissing counterparts who tended to generate more pauses.
Looking into the narrative structure of patients' speech, Daniel found that preoccupied patients took more narrative initiative whereas dismissing patients were more passive.
Interestingly, these differences between the speech patterns of dismissing and preoccupied patients were present in both modalities of therapy. This is particularly interesting considering that the two approaches differ quite substantially in their focus, with the psychoanalytic approach emphasising relationships and cognitive-behavioural therapy working more directly with symptoms.
It also suggests, that attachment-informed tailoring of psychotherapy has the potential to transcend the specifics of therapeutic modality.

One method which arguably refines the analysis of in-session verbal behaviour in relation to patient attachment, is the Patient Attachment Coding System (PACS) developed by Talia, Daniels, and Miller-Bottome \cite{Talia2014, Talia2017}.
Analysing patterns in patients' in-session behaviour and management of attunement with their therapist, Talia and colleagues \cite{Talia2014} found significant differences according to patient attachment classification.
As expected, dismissing patients showed less contact-seeking behaviour than both secure and preoccupied patients, preferring to limit emotional proximity. Somewhat surprisingly, secure patients also displayed significantly more contact-seeking behaviour than preoccupied patients, and, as expected, preoccupied patients resisted the help of therapists more than both secure and dismissing patients.

Following up, in a similar study of in-session verbal behaviour investigating the mending of ruptures in the therapeutic alliance, Miller-Bottome et al \cite{MillerBottome2018} found that securely attached patients would more openly reveal their emotions and invite the therapist to acknowledge them to resolve the rupture.
Dismissing patients, however, would minimise their own contributions and attempt to limit disclosure of their own feelings, downplaying the importance or magnitude of both affect and implied alliance ruptures.
In contrast, preoccupied patients would minimise the contributions of the therapist, limiting the attending to and resolution of alliance ruptures.
This implies that therapists should not only differentiate their approach to building the therapeutic alliance but also their management and mending of ruptures according to the attachment characteristics of their clients. The importance of this tailoring to specific challenges posed by different attachment patterns is crucial because unresolved ruptures to the therapeutic alliance can prevent therapeutic progress.

This may partly explain the findings of Dozier \cite{Dozier1990}. In this study of 40 patients with serious psychopathology, Dozier found that greater security predicted greater compliance with treatment. On the contrary, greater avoidance was associated with rejecting treatment providers, less self-disclosure in therapy, and lower ratings of the use of treatment by providers.

How therapists adjust their practice according to their patients' attachment characteristics is the topic of some research.
In a study of 12 peer-nominated experienced and highly effective therapists, Daly and Mallinckrodt \cite{Daly2009} found that the therapists strategies for management of therapeutic distance and the relationship between themselves and the client were almost mirrored over the course of treatment for clients with high attachment anxiety and avoidance.
For clients high in anxiety, therapists would begin treatment with fulfilling clients' needs for emotional proximity and then gradually increase therapeutic distance, to let the client practise managing their frustration and eventually learn to function more autonomously.
For clients high in avoidance, the corrective attachment relationship developed in the opposite direction. Here, therapists would begin by maintaining a therapeutic distance that avoidant clients found comfortable before gradually increasing proximity to let clients become comfortable with emotional closeness.
These diverging strategies suggests that experienced and effective therapists, whether intuitively or deliberately, aim to create a type of corrective attachment relationship between themselves and their clients in order to facilitate favourable change in accordance with the client's attachment characteristics.

This approach is consistent with findings suggesting that matching clients and therapists according to their activating or deactivating approaches is beneficial. Specifically, Tyrrell et al. \cite{Tyrrell1999} found that patients who were more deactivating with regards to attachment (i.e. had higher avoidance) had better therapeutic alliance ratings and functioned better with therapists who were less deactivating while the opposite was the case for patients who were less deactivating.

Findings from a study of client-nominated relationship-building incidents in early sessions of therapy confirm that clients with high avoidance prefer supportive interventions while secure clients prefer incidents with a high degree of exploration \cite{Janzen2010}.
Likewise, Petrowski et al \cite{Petrowski2011} found that patients with high attachment anxiety found therapists with higher attachment avoidance more helpful, suggesting that the naturally deactivating tendencies of the avoidant therapists were preferred over the hyper-activating or more emotionally oriented approaches of more anxious therapists.

Arietta Slade \cite{Slade2016} concludes on the basis of much of this research that the work of skilled clinicians involves adapting treatment on the basis of their patients' characteristics. While neither this idea nor the specific techniques employed by experienced therapists are new, framing them in the context of attachment can guide inexperienced therapists and experts alike in more specific ways of adjusting to the needs of their patients, and even when in the therapeutic process to do so. Namely, Slade concludes, in all cases, therapists should aim to build a strong alliance by responding "in style" with their patients early in the therapeutic process. Later, once the relationship is well established, carefully responding "out of style" can help patients develop the necessary capacities for overcoming their challenges in move beyond their mental distress. For therapists of preocuppied patients, the aim is to engage their patients emotionally, nudging them towards expression and engaging in relationships. Conversely, the aim for therapists of preoccupied patients is to help them regulate their intense emotions, gain autonomy, and learn to find comfort in the closeness they seek.

\section{Psychotherapy Research}

\chapter{Present Study}
\section{Methods}
Lorem ipsum

\subsection{Data}
Lorem ipsum

\subsection{Approach}
Lorem ipsum

\section{Results}

\section{Discussion}

\subsection{implications}

\subsection{Limitations}

\subsection{Future Work}

\section{Conclusion}


\bibliography{references}

\appendix

\end{document}
