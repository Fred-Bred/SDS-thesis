\documentclass[12pt]{report}
\usepackage{cite}
\usepackage{amsmath,amssymb,amsfonts}
\usepackage{algorithmic}
\usepackage{graphicx}
\usepackage{textcomp}
\usepackage{xcolor}
\usepackage{booktabs}
\usepackage{tabularx}
\usepackage{hyperref}

%KUstyle
% \usepackage{KUstyle}
% \ptype{Social Data Science}
% \subtitle{An Attachment-Based Therapy Metric}


% Citation style packages
%\usepackage{apacite} % Package for APA citations
%\bibliographystyle{apacite}
\bibliographystyle{plain}

% This change the content of the frontpage
\title{Anxious or Avoidant? Securing Reliable Repeated Measures of Adult Attachment Using Machine Learning}
\author{Frederik Bredgaard}
\date{May 31st 2024}

\renewcommand{\contentsname}{Table of content}

\begin{document}

\maketitle
\
\tableofcontents

\chapter{The Theory of Attachment}
Relationships are central to almost all of human life.

\section{On the Significance of Attachments}
Lorem ipsum

\subsection{Theoretical Development}
Lorem ipsum

\subsection{Clinical Significance}
Lorem ipsum

\subsection{Non-Clinical Significance}
Lorem ipsum

\section{The Measurement of Attachment}
Lorem

\subsection{The Strange Situation}
Lorem

\subsection{Adult Attachment Interview}
Lorem ipsum

\subsection{Alternatives}

\section{Patient Attachment Coding System}
Lorem ipsum \cite{Talia2017,Talia2020}

\chapter{Methods}
Lorem ipsum

\section{Data}
Lorem ipsum

\section{Approach}
Lorem ipsum

\bibliography{library}

\appendix
\chapter{Lorem ipsum}
Lorem ipsum

\chapter{Dolor}

\end{document}
