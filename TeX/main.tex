\documentclass[12pt]{report}
\usepackage{cite}
\usepackage{amsmath,amssymb,amsfonts}
\usepackage{algorithmic}
\usepackage{graphicx}
\usepackage{textcomp}
\usepackage{xcolor}
\usepackage{booktabs}
\usepackage{tabularx}
\usepackage{hyperref}

%KUstyle
% \usepackage{KUstyle}
% \ptype{Social Data Science}
% \subtitle{An Attachment-Based Therapy Metric}


% Citation style packages
%\usepackage{apacite} % Package for APA citations
%\bibliographystyle{apacite}
\bibliographystyle{plain}

% This change the content of the frontpage
\title{Anxious or Avoidant? Securing Reliable Repeated Measures of Adult Attachment Through Machine Learning}
\author{Frederik Bredgaard}
\date{May 31st 2024}

\renewcommand{\contentsname}{Table of content}

\begin{document}

\maketitle
\
\tableofcontents

\chapter{Background and Theory}
\section{Introduction}
While the development of attachment theory has continued to deliver new and valuable insights into the development of children and the relationships, pathologies, and treatments of adults through the last six decades since its inception, its clinical application has remained mostly theoretical. Arguably, this is not for a lack of clinical relevance but rather due, in large part, to the cumbersome measurement of the constructs belonging to the theory. As will be covered below, several instruments and methods for the assessment of attachment style exists. However, the gold standard for assessing adult attachment, the Adult Attachment Interview (AAI) \cite{AAITest}, is a time-consuming measure, making its application in large-scale research and clinical settings rather limited. The objective of this thesis is to develop on existing methods (e.g., the Patient Attachment Coding System \cite{Talia2017}), adding a degree of automation through language modelling approaches derived from machine learning. While the available data is limited at this stage of the field, I believe that future work can build on the approach developed here to produce statistical models for autocoding the AAI as well as a clinically relevant tool that can assist clinicians and researchers alike by making the assessment of attachment in adults more easily available and significantly more scalable.

As such, I will aim to demonstrate that automatically assessing psychotherapy patients' attachment characteristics is not only feasible but also useful in clinical and research setting, hopefully enabling future work and therapeutic applications.


\section{Attachment Theory}
This section outlines the most central elements of attachment theory from its development to the clinical relevance of different patterns of attachment and how the theory more broadly explains or mediates the effects of psychotherapy.

From this theoretical perspective, an attachment may be understood as an affectional tie formed between an individual and some other and which is characteristed by behaviours seeking to gain and maintain proximity to the other. This tie and its associated behaviours bind the two individuals together in space and endures over time \cite{Ainsworth1970}.

\subsection{Theoretical Development}
The theory of attachment is a fundamentally ethological approach, which at the time of its development sought to explain behaviours that were poorly accounted for in existing theories.

\section{The Measurement of Attachment}
Lorem

\subsection{The Strange Situation}
Lorem

\subsection{Adult Attachment Interview}
The measurement of attachment in adults is typically made using the de facto gold standard of attachment research, the Adult Attachment Interview (AAI) \cite{AAITest, Talia2019, haltigan2014adult}. The administration of the AAI takes between 60 and 90 minutes while the required verbatim transcription may take four to ten hours and coding of the transcript is expexted to take at least four hours by a coder requiring extensive training, prompting some researcher to search for simpler or less ressource-intensive methods of measurement \cite{Haas1994}.

\subsection{Patient Attachment Coding System}
Lorem ipsum \cite{Talia2017}
\section{Attachment and Therapeutic Alliance}

\section{Therapist Attunement}

\section{Psychotherapy Research}

\chapter{Methods}
Lorem ipsum

\section{Data}
Lorem ipsum

\section{Approach}
Lorem ipsum

\bibliography{references}

\appendix
\chapter{Lorem ipsum}
Lorem ipsum

\chapter{Dolor}

\end{document}
